\documentclass{beamer}

% \usepackage{beamerthemesplit} // Activate for custom appearance

\mode<presentation>
{
  \usetheme{Darmstadt}      % or try Darmstadt, Madrid, Warsaw, ...
  \usecolortheme{default} % or try albatross, beaver, crane, ...
  \usefonttheme{default}  % or try serif, structurebold, ...
  \setbeamertemplate{navigation symbols}{}
  \setbeamertemplate{caption}[numbered]
}

\title{Logic Basics}
\author{Cameron Beebe}
\institute
{
The SciPhi Initiative, LLC
\medskip
}
\date{\today}

\usepackage{graphicx}
\usepackage{amsmath}
\usepackage{amssymb}
\usepackage{booktabs}
\usepackage{tikz}
\usetikzlibrary{bayesnet}
\usetikzlibrary{arrows, decorations.markings, matrix, decorations.pathmorphing}
\usetikzlibrary{decorations.pathreplacing}
\usepackage{pgfplots}
\usepackage{pgfplotstable}
\pgfplotsset{compat=newest}
%\usepgfplotslibrary{clickable}
\usetikzlibrary{shapes}
\usetikzlibrary{plotmarks}
\usepackage{relsize}
\usepackage{natbib}
\usepackage{bussproofs}
\sloppy

\begin{document}

\frame{\titlepage}

\section[Outline]{}
\frame{\tableofcontents}


\section{Introduction}
\frame
{
\frametitle{Introduction}

\begin{itemize}
    \item<1->  Logic seems like math.
    \item<2->  Classical logic sometimes taken to be crucial aspect of foundations of math.
    \item<3->  Philosophers use to formalize arguments and sharpen critical thinking skills.
    \item<4->  BUT Philosophers and Logicians also argue about Logic itself.  Pure, Applied, Pluralism, Non-Classical, etc.
\end{itemize}


}




\section{Translating Sentences}
\frame
{
  \frametitle{Translating Sentences}

  \begin{itemize}
  \item<1->  It is \textbf{R}aining outside., The pavement is \textbf{W}et.
  \begin{itemize}
      \item<2->  \textbf{R}, \textbf{W}
  \end{itemize}
  \item<3->  It is \textbf{R}aining outside \textbf{AND} the pavement is \textbf{W}et.  It is \textbf{R}aining outside \textbf{OR} the pavement is \textbf{W}et.
  \begin{itemize}
  \item<4->  \textbf{R} \textbf{AND} \textbf{W}, \textbf{R} \textbf{OR} \textbf{W}
  \item<5->  $R \land W$, $R \lor W$
  \end{itemize}
  \item<6->  \textbf{If} it is \textbf{R}aining outside, \textbf{then} the pavement will be \textbf{W}et.
  \begin{itemize}
      \item<7->  $R \rightarrow W$
      \item<8-> Alternative notation: $R \supset W$ (Set Theory!)
  \end{itemize}  
  \item<9->  It is \textbf{not} \textbf{R}aining outside.
  \begin{itemize}
      \item<10-> $\lnot R$ 
  \end{itemize}
  \end{itemize}
}

\frame
{
\frametitle{Classical Logic}

\begin{itemize}
    \item<1-> Formally, a logic is defined over a set of terms (``alphabet'' or ``language'') by it's connectives (relational symbols), valuation functions and truth tables (semantics), and rules of inference. 
    \item<2-> Classical Logic is typically defined over a set of ``sentences'' $\mathcal{L}$ via the set of connectives $\{ \lnot, \land, \lor, \rightarrow \}$ with a binary valuation function $v(i) \mapsto \{0,1\} $ or $\{F,T\}$ for $i \in \mathcal{L}$.  Proofs are constructed according to classical logical rules of inference.
\end{itemize}


}



\section{Truth Tables}
\frame
{
\frametitle{Truth Table Examples}

\begin{center}
    
DeMorgan's Law: $\lnot (A \land B) = \lnot A \lor \lnot B$

\vspace{1cm}

\begin{tabular}{c|c|c|c|c}
$A$ & $B$ & $A\land B$ & $\lnot (A \land B)$ & $\lnot A\lor \lnot B$\\
\midrule
T & T & T & F & F \\
T & F & F & T & T \\
F & T & F & T & T \\
F & F & F & T & T 
\end{tabular}

\end{center}

}

\frame
{
\frametitle{Truth Table Examples}

\begin{center}
Material Implication: $A \rightarrow B = \lnot A \lor B$

\vspace{1cm}

\begin{tabular}{c|c|c|c}
   $A$  & $B$ & $A\to B$ & $\lnot A \lor B$ \\
   \midrule
    T & T & T & T \\
    T & F & F & F \\
    F & T & T & T \\
    F & F & T & T
\end{tabular}

\end{center}


}

\section{Validity and Soundness}

\frame
{
\frametitle{Validity and Soundness}

\begin{itemize}
    \item<1->  Validity: It is not possible for all premise formulas to be True while the conclusion is also False. (The argument's form is Truth Preserving)
    \item<2->  Soundness: All premises (of a valid argument) are actually True.
    \item<3->  Thus, it is possible for an argument to be Valid, yet not Sound.  (Most philosophizing concerns soundness.)
\end{itemize}



}



\section{Argument or Proof}

\frame
{
\frametitle{Arguments and Proofs}

\begin{itemize}
    \item<1->  Multi-line chain of formulas connected by inference rules.
    \item<2->  Proofs are `programs' according to Curry-Howard isomorphism.
    \item<3->  Important link between logic and computer science.  Algorithms.
\end{itemize}


}

\frame
{
\frametitle{Modus Ponens and Modus Tollens}

\begin{itemize}
    \item<1->
    \AxiomC{$A$}
    \AxiomC{$A \rightarrow B$}
    \BinaryInfC{$B$}
    \DisplayProof
    \item<2-> \vspace{1cm}
    \AxiomC{$A \rightarrow B$}
    \AxiomC{$\lnot B$}
    \BinaryInfC{$\lnot A$}
    \DisplayProof
    \vspace{5mm}
    \item<3-> --------- is read as ``therefore'', sometimes also denoted by $\therefore$
    \item<4-> (Plug in raining example again)
\end{itemize}



}




\section{Basic Set Theory}
\frame
{
\frametitle{Basic Set Theory}

\begin{itemize}
    \item<1-> $\subset, \supset, \cup, \cap, \{\}, \in$
    \item<2-> $\{$ stuff $\}$
    \item<3-> $\{a,b,c\} \cap \{b\} = \{b\}$
    \item<4-> $\{a,b\} \cup \{c,5\} = \{a,b,c,5\}$
    \item<5-> $\{\} \subset \{a,b,8\}$
    \item<6-> $twelve \in \{twelve, thirteen\}$
    \item<7-> Venn diagrams!
\end{itemize}

}



\end{document}